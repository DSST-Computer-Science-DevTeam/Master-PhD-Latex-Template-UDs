% Un template bilingue pour la production des m�moires et th�ses dans le d�partement de Math�matiques-Informatique.
% Ce template est conforme aux recommandations de l'�cole doctorale
%
% Ce fichier est le fichier de style principal. Vous y trouverez toutes les d�finitions des commandes personnalis�es
% Vous pouvez le modifier � votre guise pour ajouter vos effets
%
% @author Zekeng Ndadji Milliam Maxime

\usepackage[T1]{fontenc}
\usepackage[latin1]{inputenc}
\usepackage{pifont}
\usepackage{pslatex}
%\usepackage{charter}
\usepackage{mathptmx}
\usepackage{yfonts}
\usepackage[mathscr]{euscript}
\usepackage{latexsym}
\usepackage{stmaryrd}
\usepackage{amssymb}
\usepackage{amsmath}
\usepackage{graphicx}
\usepackage{pst-all}
\usepackage{verbatim} 
\usepackage{fancyvrb} %Pour utiliser l'environnement "Verbatim"
\usepackage{mathrsfs}
\usepackage{vmargin}
\usepackage{titletoc}
\usepackage{bbding}
\usepackage{wasysym}
%\usepackage{eufrak}
\usepackage{ifthen}
\usepackage[final]{pdfpages}
\usepackage{natbib}
\usepackage{morewrites}
 
\setpapersize{A4}
\DeclareMathAlphabet{\mathpzc}{OT1}{pzc}{m}{it}

\newcommand{\mathbox}[1]{\mbox{{\small \mbox{$ #1 $}}}}
\newcommand{\sta}[3]{\mathbox{#1 \stackrel{#2}{\longrightarrow} #3}}
\newcommand\QEDBox
{{\leavevmode\unskip\nobreak\hfil\penalty50\hskip.75cm%
    \hbox{} \nobreak\hfil $\Box$ \parfillskip=0pt
    \finalhyphendemerits=0
    \par
}}
\def\qed{\QEDBox}
\makeatletter
%Environnements de preuve
\newenvironment{proof}[1][{\textbf{Proof}}]{
	\par
	\normalfont
	\topsep6\p@\@plus6\p@ \trivlist
	\item[\hskip\labelsep\itshape
	#1\@addpunct{.}]\ignorespaces
}{%
	\qed\endtrivlist
}
\newenvironment{preuve}[1][{\textbf{Preuve}}]{
	\par
	\normalfont
	\topsep6\p@\@plus6\p@ \trivlist
	\item[\hskip\labelsep\itshape
	#1\@addpunct{.}]\ignorespaces
}{%
	\qed\endtrivlist
}

% *************** D�finitions de quelques couleurs ***************
\usepackage{color}
\usepackage{colortbl}

\definecolor{greenyellow}   {cmyk}{0.15, 0   , 0.69, 0   }
\definecolor{yellow}        {cmyk}{0   , 0   , 1   , 0   }
\definecolor{goldenrod}     {cmyk}{0   , 0.10, 0.84, 0   }
\definecolor{dandelion}     {cmyk}{0   , 0.29, 0.84, 0   }
\definecolor{apricot}       {cmyk}{0   , 0.32, 0.52, 0   }
\definecolor{peach}         {cmyk}{0   , 0.50, 0.70, 0   }
\definecolor{melon}         {cmyk}{0   , 0.46, 0.50, 0   }
\definecolor{yelloworange}  {cmyk}{0   , 0.42, 1   , 0   }
\definecolor{orange}        {cmyk}{0   , 0.61, 0.87, 0   }
\definecolor{burntorange}   {cmyk}{0   , 0.51, 1   , 0   }
\definecolor{bittersweet}   {cmyk}{0   , 0.75, 1   , 0.24}
\definecolor{redorange}     {cmyk}{0   , 0.77, 0.87, 0   }
\definecolor{mahogany}      {cmyk}{0   , 0.85, 0.87, 0.35}
\definecolor{maroon}        {cmyk}{0   , 0.87, 0.68, 0.32}
\definecolor{brickred}      {cmyk}{0   , 0.89, 0.94, 0.28}
\definecolor{red}           {cmyk}{0   , 1   , 1   , 0   }
\definecolor{orangered}     {cmyk}{0   , 1   , 0.50, 0   }
\definecolor{rubinered}     {cmyk}{0   , 1   , 0.13, 0   }
\definecolor{wildstrawberry}{cmyk}{0   , 0.96, 0.39, 0   }
\definecolor{salmon}        {cmyk}{0   , 0.53, 0.38, 0   }
\definecolor{carnationpink} {cmyk}{0   , 0.63, 0   , 0   }
\definecolor{magenta}       {cmyk}{0   , 1   , 0   , 0   }
\definecolor{violetred}     {cmyk}{0   , 0.81, 0   , 0   }
\definecolor{rhodamine}     {cmyk}{0   , 0.82, 0   , 0   }
\definecolor{mulberry}      {cmyk}{0.34, 0.90, 0   , 0.02}
\definecolor{redviolet}     {cmyk}{0.07, 0.90, 0   , 0.34}
\definecolor{fuchsia}       {cmyk}{0.47, 0.91, 0   , 0.08}
\definecolor{lavender}      {cmyk}{0   , 0.48, 0   , 0   }
\definecolor{thistle}       {cmyk}{0.12, 0.59, 0   , 0   }
\definecolor{orchid}        {cmyk}{0.32, 0.64, 0   , 0   }
\definecolor{darkorchid}    {cmyk}{0.40, 0.80, 0.20, 0   }
\definecolor{purple}        {cmyk}{0.45, 0.86, 0   , 0   }
\definecolor{plum}          {cmyk}{0.50, 1   , 0   , 0   }
\definecolor{violet}        {cmyk}{0.79, 0.88, 0   , 0   }
\definecolor{royalpurple}   {cmyk}{0.75, 0.90, 0   , 0   }
\definecolor{blueviolet}    {cmyk}{0.86, 0.91, 0   , 0.04}
\definecolor{periwinkle}    {cmyk}{0.57, 0.55, 0   , 0   }
\definecolor{cadetblue}     {cmyk}{0.62, 0.57, 0.23, 0   }
\definecolor{cornflowerblue}{cmyk}{0.65, 0.13, 0   , 0   }
\definecolor{midnightblue}  {cmyk}{0.98, 0.13, 0   , 0.43}
\definecolor{navyblue}      {cmyk}{0.94, 0.54, 0   , 0   }
\definecolor{royalblue}     {cmyk}{1   , 0.50, 0   , 0   }
\definecolor{blue}          {cmyk}{1   , 1   , 0   , 0   }
\definecolor{cerulean}      {cmyk}{0.94, 0.11, 0   , 0   }
\definecolor{cyan}          {cmyk}{1   , 0   , 0   , 0   }
\definecolor{processblue}   {cmyk}{0.96, 0   , 0   , 0   }
\definecolor{skyblue}       {cmyk}{0.62, 0   , 0.12, 0   }
\definecolor{turquoise}     {cmyk}{0.85, 0   , 0.20, 0   }
\definecolor{tealblue}      {cmyk}{0.86, 0   , 0.34, 0.02}
\definecolor{aquamarine}    {cmyk}{0.82, 0   , 0.30, 0   }
\definecolor{bluegreen}     {cmyk}{0.85, 0   , 0.33, 0   }
\definecolor{emerald}       {cmyk}{1   , 0   , 0.50, 0   }
\definecolor{junglegreen}   {cmyk}{0.99, 0   , 0.52, 0   }
\definecolor{seagreen}      {cmyk}{0.69, 0   , 0.50, 0   }
\definecolor{green}         {cmyk}{1   , 0   , 1   , 0   }
\definecolor{forestgreen}   {cmyk}{0.91, 0   , 0.88, 0.12}
\definecolor{pinegreen}     {cmyk}{0.92, 0   , 0.59, 0.25}
\definecolor{limegreen}     {cmyk}{0.50, 0   , 1   , 0   }
\definecolor{yellowgreen}   {cmyk}{0.44, 0   , 0.74, 0   }
\definecolor{springgreen}   {cmyk}{0.26, 0   , 0.76, 0   }
\definecolor{olivegreen}    {cmyk}{0.64, 0   , 0.95, 0.40}
\definecolor{rawsienna}     {cmyk}{0   , 0.72, 1   , 0.45}
\definecolor{sepia}         {cmyk}{0   , 0.83, 1   , 0.70}
\definecolor{brown}         {cmyk}{0   , 0.81, 1   , 0.60}
\definecolor{tan}           {cmyk}{0.14, 0.42, 0.56, 0   }
\definecolor{gray}          {cmyk}{0   , 0   , 0   , 0.50}
\definecolor{black}         {cmyk}{0   , 0   , 0   , 1   }
\definecolor{white}         {cmyk}{0   , 0   , 0   , 0   } 

% *************** Activation des liens hypertexte ***************
\ifpdf
    \pdfcompresslevel=9
        \usepackage[plainpages=false,pdfpagelabels,bookmarksnumbered,%
        colorlinks=true,%
        linkcolor=blue,%
        citecolor=blue,%
        filecolor=forestgreen,%
        pagecolor=red,%
        urlcolor=midnightblue,%
        pdftex,%
        unicode]{hyperref}
    \pdfimageresolution=600
    \usepackage{thumbpdf} 
\else
    \usepackage{hyperref}
\fi

\usepackage{memhfixc}

% *************** Nouvelle taille *******************************
\newcommand{\timesContentFontSize}{\fontsize{13pt}{18pt}}


% *************** Nouvelles Commandes ***************
\newcommand{\myChapterLabel}{Chapter}
\newcommand{\myAppendixLabel}{Appendix}
\newcommand{\lifa}{Laboratoire d'Informatique Fondamentale et Appliqu�e (LIFA)}
\newcommand{\myBibliographyTitle}{Bibliography}
\newcommand{\losname}{List of Symbols}
\newcommand{\loaname}{List of Acronyms}


% *************** Style de chapitre et de section ***************
\newcommand{\myPrintChapterLabel}[1]{
	\ifthenelse{\equal{#1}{A}}
		{\myAppendixLabel} 
		{\ifthenelse{\equal{#1}{B}}
			{\myAppendixLabel} 
			{\ifthenelse{\equal{#1}{C}}
				{\myAppendixLabel} 
				{\ifthenelse{\equal{#1}{D}}
					{\myAppendixLabel} 
					{\ifthenelse{\equal{#1}{E}}
						{\myAppendixLabel} 
						{\ifthenelse{\equal{#1}{F}}
							{\myAppendixLabel} 
							{\ifthenelse{\equal{#1}{G}}
								{\myAppendixLabel} 
								{\myChapterLabel}}}}}}} 
}

\makechapterstyle{bringhurst}{%
\renewcommand{\chapterheadstart}{}
\renewcommand{\printchaptername}{}
\renewcommand{\chapternamenum}{}
\renewcommand{\printchapternum}{}
\renewcommand{\afterchapternum}{}
\renewcommand{\printchaptertitle}[1]{%
\raggedright\Large\scshape\MakeLowercase{##1}}
\renewcommand{\afterchaptertitle}{%
\vskip\onelineskip \hrule\vskip\onelineskip}
}

\newlength{\headindent}
\newlength{\rightblock}
\makechapterstyle{southall}{%
\setlength{\headindent}{36pt}
\setlength{\rightblock}{\textwidth}
\addtolength{\rightblock}{-\headindent}
\setlength{\beforechapskip}{2\baselineskip}
\setlength{\afterchapskip}{5\baselineskip}
\setlength{\midchapskip}{0pt}
\renewcommand{\chaptitlefont}{\huge\rmfamily\raggedright}
\renewcommand{\chapnumfont}{\chaptitlefont}
\renewcommand{\printchaptername}{}
\renewcommand{\chapternamenum}{}
\renewcommand{\afterchapternum}{}
\renewcommand{\printchapternum}{%
\begin{minipage}[t][\baselineskip][b]{\headindent}
{\vspace{0pt}\chapnumfont%%%\figureversion{lining}
\thechapter}
\end{minipage}}
\renewcommand{\printchaptertitle}[1]{%
\hfill\begin{minipage}[t]{\rightblock}
{\vspace{0pt}\chaptitlefont ##1\par}\end{minipage}}
\renewcommand{\afterchaptertitle}{%
\par\vspace{\baselineskip}%
\hrulefill \par\nobreak\noindent \vskip\afterchapskip}
}

\makechapterstyle{chappell}{
\setlength\beforechapskip{0pt}
\renewcommand*\chapnamefont{\large\centering}
\renewcommand*\chapnumfont{\large}
\renewcommand*\printchapternonum{%
\vphantom{\printchaptername}%
\vphantom{\chapnumfont 1}%
\afterchapternum
\vskip -\onelineskip}
\renewcommand*\chaptitlefont{\Large\itshape}
\renewcommand*\printchaptertitle[1]{%
\hrule\vskip\onelineskip\centering\chaptitlefont ##1}
}

\makechapterstyle{fieldset}{%
    \renewcommand{\chapnamefont}{\LARGE\sffamily}%
    \renewcommand{\chapnumfont}{\fontsize{80pt}{0pt}\sffamily}%
    \renewcommand{\chaptitlefont}{\fontsize{25pt}{30pt}\Huge\bfseries}%
	% Impression du texte chapitre ou annexe
    \renewcommand{\printchaptertitle}[1]{%
		  \vspace*{-45.3pt}
		  \begin{center}
			  \chaptitlefont %\hrule height 1.5pt
			  \begin{center}\textcolor{black}{\textsc{{##1}}}\end{center}
			  \vspace{-2mm}
			  \textcolor{black}{\hrule height 2.5pt}%
		  \end{center}
        }%
		\renewcommand{\printchaptername}{%
			\vspace*{-100.3pt}
		}
	% Impression du num�ro de chapitre
	\renewcommand{\printchapternum}{%
		\begin{center}
			\chapnumfont{\textcolor{blue}{$\mathpzc{\thechapter}$}}
		\end{center}
		\vspace{0mm}
		\parbox[c]{.07\textwidth}{
			\centering
			\textcolor{black}{\hrule height 2.5pt}
		}
		\parbox[c]{.2\textwidth}{
			\centering
			\textcolor{blue}{\textsc{\myPrintChapterLabel{\thechapter}}}
		}
		\parbox[c]{.71\textwidth}{
			\centering
			\textcolor{black}{\hrule height 2.5pt}
		}
		\vspace{0mm}
	}%
}

\makechapterstyle{titleontopright}{%
    \renewcommand{\chapnamefont}{\LARGE\sffamily}%
    \renewcommand{\chapnumfont}{\fontsize{60pt}{0pt}\sffamily\bfseries}%
    \renewcommand{\chaptitlefont}{\Huge\bfseries}%
	% Impression du texte chapitre ou annexe
    \renewcommand{\printchaptertitle}[1]{%
		  \vspace*{-28.3pt}
        \chaptitlefont \hrule height 1.0pt
        \begin{flushright}\textcolor{black}{{##1}}\end{flushright}
		  \hrule height 1.0pt%
        }%
		\renewcommand{\printchaptername}{%
			\begin{flushright}
				\normalsize \chapnamefont
				$~\mathit{\myPrintChapterLabel{\thechapter}}$
			\end{flushright}
		}
	% Impression du num�ro de chapitre
    \renewcommand{\printchapternum}{%
        %\begin{flushright}
				\vspace*{-40.3pt}  
				\hspace{15,0cm}\chapnumfont{$\mathit{\thechapter}$}
		  %\end{flushright}%
        }%
}

\newcommand{\myChapterStyle}[1]{
	\chapterstyle{#1}
}

%--- Niveau 1: section
 
\newcommand{\FonteSectionI}{\sffamily\bfseries\raggedright\fontsize{16pt}{20.7pt}\selectfont}%

%\renewcommand{\thesection}{\arabic{section}}%
\renewcommand{\section}{%
   \par\vspace{20pt}
   %\hrule height 0.5mm
   \vspace{1.5mm}
   \renewcommand{\@seccntformat}[1]{\fontsize{16pt}{20.7pt}\thesection.\hspace{0.7em}}
   \@startsection{section}  % nom de l'inter
   {1}%                     % niveau de l'inter
   {0pt}%                   % l'indentation du titre et du texte suivant
   {4pt}% beforeskip %
   {6pt}% afterskip
   {\FonteSectionI}%        % style
}

%--- Niveau 2: sous-section
\newcommand{\FonteSectionII}{\sffamily\bfseries\raggedright\fontsize{15pt}{19.4pt}\selectfont}%

\renewcommand{\thesubsection}{\thesection.\arabic{subsection}}%
\renewcommand{\subsection}{%
\vspace{3mm}
  \renewcommand{\@seccntformat}[1]%
               {{\fontsize{15pt}{19.4pt}\thesubsection.\hspace{0.7em}}}%
  \@startsection%
   {subsection}%            % nom de l'inter
   {2}%                     % niveau de l'inter
   {0pt}%                   % l'indentation du titre et du texte suivant
   {3pt}
   {5pt}
   {\FonteSectionII}}%      % style

%--- Niveau 3: sous-sous-section 

\newcommand{\FonteSectionIII}{\sffamily\bfseries\fontsize{14pt}{18.2pt}\raggedright\selectfont}%

\renewcommand{\thesubsubsection}{\thesubsection.\arabic{subsubsection}}%
\renewcommand{\subsubsection}{%
\vspace{2mm}
  \renewcommand{\@seccntformat}[1]{%
               {\sffamily\bfseries\fontsize{14pt}{18.2pt}\thesubsubsection.\hspace{0.7em}}}%
  \@startsection%
   {subsubsection}%         % nom de l'inter
   {3}%                     % niveau de l'inter
   {0pt}%                   % l'indentation du titre et du texte suivant
   {3pt}
   {3pt}
   {\FonteSectionIII}}%     % style


% <Alin�as>----------------------------------------------------------------

% Disable single lines at the start of a paragraph
\clubpenalty = 10000
% Disable single lines at the end of a paragraph 
\widowpenalty = 10000
\displaywidowpenalty = 10000

% Ent�te et pied de page
\newcommand{\doctypethesis}{Th�se de Doctorat en}
\newcommand{\doctypemaster}{M�moire de Master en}
\newcommand{\doctype}{\ifthenelse{\equal{\doclevel}{\master}}{\doctypemaster}{\doctypethesis}}
\newcommand{\phd}{PhD}
\newcommand{\master}{Master}
\newcommand{\doclevel}{\phd}
\newcommand{\level}[1]{
	\renewcommand{\doclevel}{#1}
}
\newcommand{\phdthesislabel}{\doctype $~$ \studentspeciality $~$, Universit� de Dschang}
\newcommand{\studentspeciality}{\computerScience}
\newcommand{\speciality}[1]{
	\renewcommand{\studentspeciality}{#1}
}
\newcommand{\computerScience}{Informatique}
\newcommand{\mathematics}{Math�matiques}
\newcommand{\studentlab}{LIFA}
\newcommand{\lab}[1]{
	\renewcommand{\studentlab}{#1}
}

\makeevenhead{ruled}{\small\textsc{\rightmark}}{}{\thepage}
\makeoddhead{ruled}{\small\textsc{\rightmark}}{}{\thepage}
\makeoddfoot{ruled}{\hrule height 0.3mm \small\textsc{\phdthesislabel}}{}{\small\textsc{\studentlab}}
\makeevenfoot{ruled}{\hrule height 0.3mm \small\textsc{\phdthesislabel}}{}{\small\textsc{\studentlab}}

% *************** Style table de mati�re et listes de figures ***************
\settocdepth{subsubsection}
\setsecnumdepth{subsubsection}
\maxsecnumdepth{subsubsection}
\settocdepth{subsubsection}
\maxtocdepth{subsubsection}

\renewcommand{\chapternumberline}[1]{
% Impression du texte chapitre ou annexe
\hspace{-0.4cm}\textbf{
	$\mathit{\myPrintChapterLabel{#1}}$
	$\mathpzc{{#1}}~\RHD$}
}

\newcommand{\lof}{false}
\renewcommand{\numberline}[1]{
\ifthenelse{\equal{\lof}{false}}{\hspace{-0.6cm}$\mathrm{{#1}}$ -} {$\mathrm{{#1}}$ -}
}

\let\oldcontentsline=\contentsline
\renewcommand{\contentsline}[4]{
	\vspace{0.8mm}
	\oldcontentsline{#1}{#2}{#3}{#4}
	\vspace{0.8mm}
}

% Notes de bas de page
%\newcommand{\FonteNoteBasPage}{\footnotesize\sffamily}
%\renewcommand{\footnotesize}{\FonteNoteBasPage}
\addtolength{\skip\footins}{6pt} 

\renewcommand{\footnoterule}{%
	\par %\vspace*{-12.3pt}
	\noindent\rule{3.5cm}{0.6pt}\vspace*{6pt} 
}

\setlength{\footnotesep}{3pt} % Espace vertical avant chaque note (strut)

\newcommand{\@Myfnmark}{
      \mbox{\fontsize{8}{11}\sffamily\arabic{footnote}. }%
}

\renewcommand{\@makefntext}[1]{%
      \noindent\@Myfnmark#1%
}%

\def\@thefnmark{\arabic{footnote}}


% Environnement personnalis� de description
\newcommand{\myDescription}[2]{
\par\vspace{0.35cm}\noindent\textbf{#1}
\begin{list}{}{}
	\item \noindent #2
\end{list}
\vspace{2pt}
}

\newcommand{\minitoclevel}{section}
\newcommand{\minitocstyle}{titleontopright}

% Insertion d'une mini table de mati�re
\newcommand{\myMiniToc}[2]{
	\ifthenelse{\equal{#1}{}}
	{\renewcommand{\minitoclevel}{section}}
	{\renewcommand{\minitoclevel}{#1}}
	\startcontents[chapters]
	\ifthenelse{\equal{\minitocstyle}{fieldset}}
	{\myMiniTocFieldset{#1}{#2}}
	{
		\ifthenelse{\equal{\minitocstyle}{titleontopright}}
		{\myMiniTocTitleOnTopRight{#1}{#2}}
		{}
	}
}

% Minitoc de type titleontop
\newcommand{\myMiniTocTitleOnTopRight}[2]{
	\vspace{-0.7mm}
	\vspace{20pt}
	%\hspace{-22pt}
	\begin{minipage}{15cm}
	\begin{flushright}\noindent\textcolor{black}{\textbf{#2}} \vspace{5pt} \hrule height 0.08mm \end{flushright}
	\par
	\printcontents[chapters]{}{1}{}
	\par
	\begin{flushright} \vspace{0pt} \hrule height 0.08mm \vspace{30pt} \end{flushright}
	\end{minipage}
}

% Minitoc de type fieldset
\newcommand{\myMiniTocFieldset}[2]{
	%\vspace{-0.7mm}
	\vspace{5pt}
	%\hspace{-22pt}
	\begin{minipage}{15cm}
		\parbox[c]{.07\textwidth}{
			\centering
			\textcolor{black}{\hrule height 0.4mm}
		}
		\parbox[c]{.2\textwidth}{
			\centering
			\textcolor{black}{\textsc{\textbf{#2}}}
		}
		\parbox[c]{.71\textwidth}{
			\centering
			\textcolor{black}{\hrule height 0.4mm}
		}
		%\vspace{-10pt} 
		\par
		\printcontents[chapters]{}{1}{}
		\par
		\begin{flushright} 
			\vspace{0pt} 
			\hrule height 0.4mm 
			\vspace{30pt} 
		\end{flushright}
	\end{minipage}
}

\newcommand{\myMiniTocClearPage}[2]{
	\myMiniToc{#1}{#2}
	\clearpage
}

\newcommand{\myMiniTocStyle}[1]{
	\renewcommand{\minitocstyle}{#1}
}

\newcommand{\myBibliography}[2]{
	\bibliographystyle{#1}
	\bibliography{#2}
	\myCleanStarChapterEnd
}

% Bibligraphie
\renewenvironment{thebibliography}[1]{%BEGIN
   \myChapterStar{\myBibliographyTitle}{}{true}\label{biblio}%
   \begin{myBiblio}
  }{%END
   \end{myBiblio}
}

\def\bibi[#1]{\item[\@biblabel{#1}\hfill]} % @ special
\newenvironment{myBiblio}{%BEGIN
   \list{}{
         \usecounter{enumiv}%
         \let\p@enumiv\@empty
         \renewcommand\theenumiv{\arabic{enumiv}}%
         \renewcommand\newblock{\hskip .11em \@plus.33em \@minus.07em}%
         %% dimensions horizontales
         \setlength{\leftmargin}{0mm}%%%
         %\setlength{\itemindent}{-3mm}%%%
         \setlength{\labelsep}{2mm}%%%
         \setlength{\labelwidth}{10mm}%%%
         %% dimensions verticales
          \setlength{\topsep}{0pt}%
          \setlength{\parskip}{6pt}%
          \setlength{\itemsep}{5pt}%
          \setlength{\partopsep}{0pt}%
          \setlength{\parsep}{3pt}%
         \sloppy\clubpenalty4000\widowpenalty4000%
         \sfcode`\.=\@m
         }%
  }{%END
      \def\@noitemerr{\@latex@warning{Empty 'thebibliography' environment}}
      %\FonteTexte%
      \endlist%
}

% Tableaux
\usepackage[format=hang,font=large,labelfont=bf,textfont=it,skip=5pt,labelsep=endash]{caption}
\captionsetup[table]{name=Table,position=top}
\captionsetup[figure]{name=Figure,position=bottom}
\renewcommand{\thetable}{\Roman{table}}
\newcommand{\tocsetted}{false}

% Quelques raccourcis utiles
\newcommand{\myTableOfContents}[1]{
	\ifthenelse{\equal{\tocsetted}{false}}
	{\clearpage}{}
	\mySaveMarks
	\ifthenelse{\equal{#1}{}}{}
	{\renewcommand{\contentsname}{#1}}
	\addcontentsline{toc}{section}{\myNumberLine{\contentsname}}
	\renewcommand{\leftmark}{\contentsname}
	\renewcommand{\rightmark}{\contentsname}
	\tableofcontents*
	\myCleanStarChapterEnd
	\renewcommand{\tocsetted}{true}
}

\newcommand{\myTableOfContentsStar}[1]{
	\ifthenelse{\equal{\tocsetted}{false}}
	{\clearpage}{}
	\mySaveMarks
	\ifthenelse{\equal{#1}{}}{}
	{\renewcommand{\contentsname}{#1}}
	\renewcommand{\leftmark}{\contentsname}
	\renewcommand{\rightmark}{\contentsname}
	\tableofcontents*
	\myCleanStarChapterEnd
	\renewcommand{\tocsetted}{true}
}

\newcommand{\myListOfSymbols}[1]{
	\ifthenelse{\equal{#1}{}}{}
	{\renewcommand{\losname}{#1}}
	\myChapterStar{\losname}{}{section}
	\begin{center}
	\begin{tabular}[t]{rp{5mm}p{12cm}}
		$\mathbb{G}$ & &  A grammatical model of workflow; \\
		$t_{i_f}$ & & A global artefact obtained after merging a set of artefacts.
	\end{tabular}
\end{center}

	\myCleanStarChapterEnd
	\renewcommand{\tocsetted}{true}
}

\newcommand{\myListOfSymbolsStar}[1]{
	\ifthenelse{\equal{#1}{}}{}
	{\renewcommand{\losname}{#1}}
	\myChapterStar{\losname}{}{false}
	\begin{center}
	\begin{tabular}[t]{rp{5mm}p{12cm}}
		$\mathbb{G}$ & &  A grammatical model of workflow; \\
		$t_{i_f}$ & & A global artefact obtained after merging a set of artefacts.
	\end{tabular}
\end{center}

	\myCleanStarChapterEnd
	\renewcommand{\tocsetted}{true}
}

\newcommand{\myListOfAcronyms}[1]{
	\ifthenelse{\equal{#1}{}}{}
	{\renewcommand{\loaname}{#1}}
	\myChapterStar{\loaname}{}{section}
	\begin{center}
	\begin{tabular}[t]{rp{5mm}p{12cm}}
		P2P & & Peer to Peer; \\
		BPMN & & Business Process Model and Notation.
	\end{tabular}
\end{center}

	\myCleanStarChapterEnd
	\renewcommand{\tocsetted}{true}
}

\newcommand{\myListOfAcronymsStar}[1]{
	\ifthenelse{\equal{#1}{}}{}
	{\renewcommand{\loaname}{#1}}
	\myChapterStar{\loaname}{}{false}
	\begin{center}
	\begin{tabular}[t]{rp{5mm}p{12cm}}
		P2P & & Peer to Peer; \\
		BPMN & & Business Process Model and Notation.
	\end{tabular}
\end{center}

	\myCleanStarChapterEnd
	\renewcommand{\tocsetted}{true}
}

\newcommand{\myListOfFigures}[1]{
	\ifthenelse{\equal{\tocsetted}{false}}
	{\clearpage}{}
	\mySaveMarks
	\ifthenelse{\equal{#1}{}}{}
	{\renewcommand{\listfigurename}{#1}}
	\addcontentsline{toc}{section}{\myNumberLine{\listfigurename}}
	\renewcommand{\leftmark}{\listfigurename}
	\renewcommand{\rightmark}{\listfigurename}
	\renewcommand{\lof}{true}
	\listoffigures*
	\renewcommand{\lof}{false}
	\myCleanStarChapterEnd
	\renewcommand{\tocsetted}{true}
}

\newcommand{\myListOfFiguresStar}[1]{
	\ifthenelse{\equal{\tocsetted}{false}}
	{\clearpage}{}
	\mySaveMarks
	\ifthenelse{\equal{#1}{}}{}
	{\renewcommand{\listfigurename}{#1}}
	\renewcommand{\leftmark}{\listfigurename}
	\renewcommand{\rightmark}{\listfigurename}
	\renewcommand{\lof}{true}
	\listoffigures*
	\renewcommand{\lof}{false}
	\myCleanStarChapterEnd
	\renewcommand{\tocsetted}{true}
}

\newcommand{\myListOfTables}[1]{
	\ifthenelse{\equal{\tocsetted}{false}}
	{\clearpage}{}
	\mySaveMarks
	\ifthenelse{\equal{#1}{}}{}
	{\renewcommand{\listtablename}{#1}}
	\addcontentsline{toc}{section}{\myNumberLine{\listtablename}}
	\renewcommand{\leftmark}{\listtablename}
	\renewcommand{\rightmark}{\listtablename}
	\renewcommand{\lof}{true}
	\listoftables*
	\renewcommand{\lof}{false}
	\myCleanStarChapterEnd
	\renewcommand{\tocsetted}{true}
}

\newcommand{\myListOfTablesStar}[1]{
	\ifthenelse{\equal{\tocsetted}{false}}
	{\clearpage}{}
	\mySaveMarks
	\ifthenelse{\equal{#1}{}}{}
	{\renewcommand{\listtablename}{#1}}
	\renewcommand{\leftmark}{\listtablename}
	\renewcommand{\rightmark}{\listtablename}
	\renewcommand{\lof}{true}
	\listoftables*
	\renewcommand{\lof}{false}
	\myCleanStarChapterEnd
	\renewcommand{\tocsetted}{true}
}

\newcommand{\myChapter}[2]{
	\chapter[#2]{#1}
}

\newcommand{\shortTitle}{}

\newcommand{\mySaveMarks}{
	\let\oldleftmark=\leftmark
	\let\oldrightmark=\rightmark
}

\newcommand{\myNumberLine}[1]{
	\hspace{-0.55cm}#1
}

\newcommand{\myChapterNumberLine}[1]{
	\hspace{-0.25cm}#1
}

\newcommand{\myChapterStar}[3]{
	\mySaveMarks
	\ifthenelse{\equal{#2}{}}
	{\renewcommand{\shortTitle}{#1}}
	{\renewcommand{\shortTitle}{#2}}
	\renewcommand{\leftmark}{\shortTitle}
	\renewcommand{\rightmark}{\shortTitle}
	\chapter*{#1}
	\ifthenelse{\equal{#3}{false}}
	{}
	{
		\ifthenelse{\equal{#3}{}}
		{\addcontentsline{toc}{chapter}{\myChapterNumberLine{\shortTitle}}}
		{
			\ifthenelse{\equal{#3}{true}}
			{\addcontentsline{toc}{chapter}{\myChapterNumberLine{\shortTitle}}}
			{
				\ifthenelse{\equal{#3}{chapter}}
				{\addcontentsline{toc}{#3}{\myChapterNumberLine{\shortTitle}}}
				{\addcontentsline{toc}{#3}{\myNumberLine{\shortTitle}}}
			}
		}
	}
}

\newcommand{\mySection}[2]{
	\resumecontents[chapters]
	\ifthenelse{\equal{#2}{}}
	{\section{#1}}
	{\section[#2]{#1}}
	\hrule height 0.5mm
	\vspace{5mm}
	\stopcontents[chapters]
}

\newcommand{\mySectionStar}[3]{
	\resumecontents[chapters]
	\ifthenelse{\equal{#2}{}}
	{\renewcommand{\shortTitle}{#1}}
	{\renewcommand{\shortTitle}{#2}}
	\renewcommand{\rightmark}{\shortTitle}
	\section*{#1}
	\hrule height 0.5mm
	\vspace{5mm}
	\ifthenelse{\equal{#3}{false}}
	{}
	{
		\ifthenelse{\equal{#3}{}}
		{\addcontentsline{toc}{section}{\myNumberLine{\shortTitle}}}
		{
			\ifthenelse{\equal{#3}{true}}
			{\addcontentsline{toc}{section}{\myNumberLine{\shortTitle}}}
			{
				\ifthenelse{\equal{#3}{chapter}}
				{\addcontentsline{toc}{#3}{\myChapterNumberLine{\shortTitle}}}
				{\addcontentsline{toc}{#3}{\myNumberLine{\shortTitle}}}
			}
		}
	}
	\stopcontents[chapters]
}

\newcommand{\mySubSection}[2]{
	\ifthenelse{\equal{\minitoclevel}{section}}{}
	{\resumecontents[chapters]}
	\ifthenelse{\equal{#2}{}}
	{\subsection{#1}}
	{\subsection[#2]{#1}}
	\stopcontents[chapters]
}

\newcommand{\mySubSectionStar}[3]{
	\ifthenelse{\equal{\minitoclevel}{section}}{}
	{\resumecontents[chapters]}
	\ifthenelse{\equal{#2}{}}
	{\renewcommand{\shortTitle}{#1}}
	{\renewcommand{\shortTitle}{#2}}
	\renewcommand{\rightmark}{\shortTitle}
	\subsection*{#1}
	\ifthenelse{\equal{#3}{false}}
	{}
	{
		\ifthenelse{\equal{#3}{}}
		{\addcontentsline{toc}{subsection}{\myNumberLine{\shortTitle}}}
		{
			\ifthenelse{\equal{#3}{true}}
			{\addcontentsline{toc}{subsection}{\myNumberLine{\shortTitle}}}
			{
				\ifthenelse{\equal{#3}{chapter}}
				{\addcontentsline{toc}{#3}{\myChapterNumberLine{\shortTitle}}}
				{\addcontentsline{toc}{#3}{\myNumberLine{\shortTitle}}}
			}
		}
	}
	\stopcontents[chapters]
}

\newcommand{\mySubSubSection}[2]{
	\ifthenelse{\equal{\minitoclevel}{subsubsection}}
	{\resumecontents[chapters]}{}
	\ifthenelse{\equal{#2}{}}
	{\subsubsection{#1}}
	{\subsubsection[#2]{#1}}
	\stopcontents[chapters]
}

\newcommand{\mySubSubSectionStar}[3]{
	\ifthenelse{\equal{\minitoclevel}{subsubsection}}
	{\resumecontents[chapters]}{}
	\ifthenelse{\equal{#2}{}}
	{\renewcommand{\shortTitle}{#1}}
	{\renewcommand{\shortTitle}{#2}}
	\renewcommand{\rightmark}{\shortTitle}
	\subsubsection*{#1}
	\ifthenelse{\equal{#3}{false}}
	{}
	{
		\ifthenelse{\equal{#3}{}}
		{\addcontentsline{toc}{subsubsection}{\myNumberLine{\shortTitle}}}
		{
			\ifthenelse{\equal{#3}{true}}
			{\addcontentsline{toc}{subsubsection}{\myNumberLine{\shortTitle}}}
			{
				\ifthenelse{\equal{#3}{chapter}}
				{\addcontentsline{toc}{#3}{\myChapterNumberLine{\shortTitle}}}
				{\addcontentsline{toc}{#3}{\myNumberLine{\shortTitle}}}
			}
		}
	}
	\stopcontents[chapters]
}

\newcommand{\myRestoreMarks}{
	\let\leftmark=\oldleftmark
	\let\rightmark=\oldrightmark
}

\newcommand{\myCleanStarChapterEnd}{
	\clearpage
	\myRestoreMarks
}

\newcommand{\currentlanguage}{english}

\newcommand{\switchLanguage}[1]{
	\renewcommand{\currentlanguage}{#1}
	\ifthenelse{\equal{#1}{fran�ais}}
	{
		\usepackage[frenchb]{babel}
		\renewcommand{\myChapterLabel}{Chapitre}
		\renewcommand{\myAppendixLabel}{Annexe}
		\renewcommand{\lifa}{Laboratoire d'Informatique Fondamentale et Appliqu�e (LIFA)}
		\renewcommand{\myBibliographyTitle}{Bibliographie}
		\renewcommand{\phdthesislabel}{Th�se de Doctorat en $~$ \studentspeciality $~$, Universit� de Dschang}
		\renewcommand{\computerScience}{Informatique}
		\renewcommand{\mathematics}{Math�matiques}
		\renewcommand{\studentlab}{LIFA}
		\renewcommand{\doctypethesis}{Th�se de Doctorat en}
		\renewcommand{\doctypemaster}{M�moire de Master en}
		\renewcommand{\losname}{Liste des Symboles}
		\renewcommand{\loaname}{Liste des Acronymes}
	}{
		\usepackage[english]{babel}
		\renewcommand{\myChapterLabel}{Chapter}
		\renewcommand{\myAppendixLabel}{Appendix}
		\renewcommand{\lifa}{Laboratoire d'Informatique Fondamentale et Appliqu�e (LIFA)}
		\renewcommand{\myBibliographyTitle}{Bibliography}
				\renewcommand{\phdthesislabel}{PhD Thesis in $~$ \studentspeciality $~$, University of Dschang}
		\renewcommand{\computerScience}{Computer Science}
		\renewcommand{\mathematics}{Mathematics}
		\renewcommand{\studentlab}{LIFA}
		\renewcommand{\doctypethesis}{PhD Thesis in}
		\renewcommand{\doctypemaster}{Master Report in}
		\renewcommand{\losname}{List of Symbols}
		\renewcommand{\loaname}{List of Acronyms}
	}
}

\letcountercounter{sidefootnote}{footnote}
