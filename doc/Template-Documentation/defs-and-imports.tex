% Un template bilingue pour la production des m�moires et th�ses dans le d�partement de Math�matiques-Informatique.
% Ce template est conforme aux recommandations de l'�cole doctorale
%
% Ce fichier est con�u pour accueillir vos imports (\usepackage) et vos propres d�finitions d'environnements latex et/ou de style
% Consulter le fichier style.tex pour savoir ce qui a d�ja �t� import�
%
% @author Zekeng Ndadji Milliam Maxime

\usepackage[english]{algorithm2e} %Pour �crire des algorithmes
%\SetKwIF{Si}{SinonSi}{Sinon}{si}{alors}{sinon si}{sinon}{finsi} %pour franciser le If
%\SetKw{Debut}{Fin}
\SetKwFor{For}{for}{do}{endfor}
%\SetKwFor{PourTout}{pourTout}{faire}{finPour}%pour franciser le pour (on a remplacer ''pour'' par ''pourTout'' 
%\SetKwForAll{PourTout}{pourTout}{faire}{finPour}
\SetKwRepeat{Repeat}{repeat}{until}
%\SetKwRepeat{Repeter}{repeter}{jusqu'�}%pour franciser le ''repeter''
%\restylealgo{boxed}\linesnumbered %pour encader les algorithmes et num�roter ses lignes
%\SetKwFor{Tantque}{tantque}{faire}{fintq}%pour franciser le tant que


%D�finition de nouveaux environnements de type th�or�me
\newtheorem{theorem}{Th�or�me}
\newtheorem{definition}[theorem]{D�finition}
\newtheorem{proposition}[theorem]{Proposition}
\newtheorem{lemma}[theorem]{Lemme}
\newtheorem{example}[theorem]{Exemple}
\newtheorem{remark}[theorem]{Remarque}
\newtheorem{corollary}[theorem]{Corollaire}
\newtheorem{problem}[theorem]{Probl�me}

