\newcommand{\myBibliography}[2]{
	\bibliographystyle{#1}
	\bibliography{#2}
	\myCleanStarChapterEnd
}

% Bibligraphie
\renewenvironment{thebibliography}[1]{%BEGIN
   \myChapterStar{\myBibliographyTitle}{}{true}\label{biblio}%
   \begin{myBiblio}
  }{%END
   \end{myBiblio}
}

\def\bibi[#1]{\item[\@biblabel{#1}\hfill]} % @ special
\newenvironment{myBiblio}{%BEGIN
   \list{}{
         \usecounter{enumiv}%
         \let\p@enumiv\@empty
         \renewcommand\theenumiv{\arabic{enumiv}}%
         \renewcommand\newblock{\hskip .11em \@plus.33em \@minus.07em}%
         %% dimensions horizontales
         \setlength{\leftmargin}{0mm}%%%
         %\setlength{\itemindent}{-3mm}%%%
         \setlength{\labelsep}{2mm}%%%
         \setlength{\labelwidth}{10mm}%%%
         %% dimensions verticales
          \setlength{\topsep}{0pt}%
          \setlength{\parskip}{6pt}%
          \setlength{\itemsep}{5pt}%
          \setlength{\partopsep}{0pt}%
          \setlength{\parsep}{3pt}%
         \sloppy\clubpenalty4000\widowpenalty4000%
         \sfcode`\.=\@m
         }%
  }{%END
      \def\@noitemerr{\@latex@warning{Empty 'thebibliography' environment}}
      %\FonteTexte%
      \endlist%
}

% Tableaux
\usepackage[format=hang,font=large,labelfont=bf,textfont=it,skip=5pt,labelsep=endash]{caption}
\captionsetup[table]{name=Table,position=top}
\captionsetup[figure]{name=Figure,position=bottom}
\newcommand{\tocsetted}{false}

% Des red�finitions suppl�mentaires
\let\oldmainmatter=\mainmatter
\renewcommand{\mainmatter}{
	\oldmainmatter
	% Num�roter les chapitres en chiffres romains
	\renewcommand{\thechapter}{\Roman{chapter}}
	% Numeroter les tableaux en chiffres romains
	\renewcommand{\thetable}{\thechapter.\Roman{table}}
}

\let\oldappendix=\appendix
\renewcommand{\appendix}{
	\oldappendix
	% Numeroter les tableaux en chiffres romains
	\renewcommand{\thetable}{\thechapter.\Roman{table}}
}
